\documentclass[]{article}
\usepackage[utf8]{inputenc}
\usepackage[backend=biber]{biblatex}
\usepackage{titlesec}
\addbibresource{case.bib}

\titleformat{\paragraph}
{\normalfont\normalsize\bfseries}{\theparagraph}{1em}{}
\titlespacing*{\paragraph}
{0pt}{3.25ex plus 1ex minus .2ex}{1.5ex plus .2ex}

\titleformat{\subparagraph}
{\normalfont\footnotesize\bfseries}{\thesubparagraph}{1em}{}
\titlespacing*{\subparagraph}
{0pt}{3.25ex plus 1ex minus .2ex}{1.5ex plus .2ex}

%opening
\title{Hoe combineer je PRINCE2 of PMBOK met agile werken?}
\author{Thema 4: Van Loo Lieven, Van Looy Robbe, Storme Flor, Zwartjes Tijl}

\begin{document}

\maketitle
\begin{abstract}

\end{abstract}

\tableofcontents


\section{LiteratuurStudie}
\subsection{Samenvatting}

\subsubsection{PRINCE2}
PRINCE2 is een methode voor projectmanagement. De methode is ontwikkeld door de Britse semioverheidsorganisatie Office Of Government Commerce. De PRINCE2-methode omvat 4 elementen: principes, thema’s, processen en de projectomgeving. Er zijn 7 principes en deze bestaan uit verplichtingen en best practices. Alle principes moeten worden gevolgd, anders wordt het project niet beschouwt als een PRINCE2 project.  Dit zijn dus de onderliggende regels die er zijn die je automatisch doet. Als deze principes niet juist worden toegepast, voelt dit raar. De thema’s worden voortdurend behandeld doorheen het project. Deze hebben we nodig doorheen het project, maar kunnen tijdens het project veranderen. De processen zijn de stapsgewijze vooruitgang doorheen een project. Ze vertellen wat er moet gebeuren met de thema’s op een bepaald punt in het project, niet vergeten, gebruikmakend van de principes. Bij het laatste element kijkt men naar de omgeving, de grootte, de complexiteit van het project en er zeker van zijn dat het goed bij het project past.

\paragraph{Principes}

\subparagraph{Zakelijke rechtvaardiging}
Er moet een gerechtvaardigde reden zijn om het project door te zetten. Als er geen reden is, dan moet het project afgesloten worden.

\subparagraph{Van ervaring leren}
Om een project in goede banen te leiden, moet er zich gebaseerd worden op ervaringen van vorige projecten. 

\subparagraph{Gedefinieerde rollen en verantwoordelijkheden}
Het projectteam moet een duidelijke organisatiestructuur bevatten. Iedereen moet z’n rol en de corresponderende verantwoordelijkheid kennen.

\subparagraph{Gedefinieerde rollen en verantwoordelijkheden}
Een PRINCE2 project heeft vaste rollen, deze zijn:
\begin{itemize}
	\item Project Board \\ Deze zijn de 3 primaire stakeholders.
	\item Project Assurance \\ Zij verzekeren de belangen van de stakeholders.
	\item Change Authority \\ Maken beslissingen over change in naam van de project board.
	\item Project Manager \\ Staat in voor alledaagse management van het project.
	\item Project Support \\ Helpt de project manager.
	\item Team Manager \\ Garandeert de efficiëntie van de productie van het team.
	
\end{itemize}

\subparagraph{Beheer in fasen}
Complexe taken moeten worden opgesplitst in afzonderlijke fasen, dit worden management fasen genoemd. Een PRINCE2 project wordt fase per fase uitgedacht, gemonitord en gecontroleerd. De fasen worden gescheiden door beslissingspunten door het projectmanagement. Aan het einde van elke fase wordt er beslist op basis van de resultaten van de voorgaande fasen of men al dan niet mag beginnen aan de volgende fase. Dit is de keuze van het projectboard, dit heeft een grotere controle over het project en dus meer controle voor het senior management. Enkele voordelen van het faseren van een project:
\begin{itemize}
\item Het project verdelen in minder complexe delen.
\item Een gedetailleerd fase plan.
\item Zorgt voor leerrijke omgevingen voor volgende fasen.
\end{itemize}

\subparagraph{Beheer in uitzondering}
Dit wordt gebruikt om de ondergeschikte laag te besturen. Als er een groot probleem zich voordoet, een ‘Exception’ of ‘Uitzondering’, dan wilt dit zeggen dat het probleem boven de vastgelegde tolerantiegrens gaat. Een lid van het projectboard zal nooit iets horen van de projectmanager als het probleem de tolerantiegrens niet overschrijd.

\subparagraph{Zich richten op de producten}
Een goede productomschrijving is zeer belangrijk tijdens het leiden van het project. Elke stakeholder moet hetzelfde product in gedachten hebben. De productomschrijving moet zo snel en duidelijk mogelijk worden geconstrueerd.

\subparagraph{Zich richten op de producten}
Een goede productomschrijving is zeer belangrijk tijdens het leiden van het project. Elke stakeholder moet hetzelfde product in gedachten hebben. De productomschrijving moet zo snel en duidelijk mogelijk worden geconstrueerd.

\subparagraph{Op maat van de projectomgeving}
Een PRINCE2 project moet op maat gemaakt zijn van het project op vlak van complexiteit, risico, omgeving, …. PRINCE2 is gemaakt voor alle types van projecten, maar maakt snel gemakkelijke projecten moeilijk. Daarom is het belangrijk om te weten wanneer je PRINCE2 gebruikt of een versimpelde versie.


\paragraph{Thema's}
\subparagraph{Business Case}
De business case biedt een structuur om te beoordelen of het project wenselijk, levensvatbaar en haalbaar is en de investering die tijdens het project wordt gedaan, waard is. De business case is iets wat evolueert doorheen het project. 

Bij de start van het project wordt er een outline business case opgesteld, maar als we verdergaan door de initiatiefase wordt deze echter geüpdatet naar een volledig gedetailleerde business case. Na elke fase wordt er dan gekeken of er nog rechtvaardiging(zie principes) is om het project naar de volgende fase te laten gaan.

\subparagraph{Wijziging}
Wijziging is niet alleen kunnen omgaan met veranderingen in de planning die komen door beslissingen, maar ook het kunnen inschatten en voorbereiden op wijzigingen die moeten komen door problemen die opduiken. Dit gaat natuurlijk hand-in-hand met risk management.
[Risk management uitleggen]--------------

Eén van de belangrijkste eigenschappen van een project manager is het kunnen weigeren of accepteren van voorgestelde wijzigingen. En als vaak vergeten deel, het bedanken van de persoon die het heeft voorgesteld.

Prince 2 heeft een vaste aanpak voor het documenteren van een wijziging, dit omvat 6 zaken:
\begin{itemize}
	\item Change Control Approach \\ Dat beschrijft hoe issues en wijzigingen aangepakt worden in het project.
	\item Configuration Item Record \\ Dit is een dataset voor elk product in het project.
	\item Product Status Account \\ Dit is een statusreport van elk product in het project.
	\item Daily Log \\ Hierin wordt door de project manager alle kleine informele zaken genoteerd.
	\item Issue Register \\ Dit is het formeel document voor de project manager.
	\item Issue Report \\ Hier wordt de issue in detail uitgelegd. Dit kan volgens PRINCE2 3 vormen hebben:Vraag voor wijziging, off-specification of probleem/vrees.
\end{itemize}

\subparagraph{Organisatie}
Dit staat voor alle mensen die men nodig heeft op een project. Belangrijkste van al is de Corporate- of program management. Zij staan er in om het project te initialiseren.

In het project management team heb je: een project board die het project stuurt, project manager die het project beheert in naam van het project board en de team manager die instaat voor het leveren van de feature products.

\subparagraph{Plannen}
Bij het plannen van een project wordt er meestal gebruik gemaakt van vorige projecten of soortgelijke projecten. Men kan aan de hand van die projecten het plan opstellen voor het huidig project. 

Eén van de eerste zaken die men moet proberen verkrijgen tijdens het plannen is een idee van de scope, een product breakdown structuur kan helpen en zorgt ervoor dat iedereen het plan snel begrijpt. 

\subparagraph{Voortgang}
Voortgang gaat vooral over hoe het project wordt gecontroleerd, waar de voortgang op het huidige moment zit en dit vergelijken met het plan dat werd opgesteld. Het voortgang thema kan uitgelegd worden aan de hand van drie delen. 

Het in het oog houden van de vooruitgang met de voorop geplande tijdsperiodes. Kunnen voorspellen, hoelang men ergens aan gaat bezig zijn en correct kunnen handelen als er een onverwacht obstakel zich voordoet.

\subparagraph{Kwaliteit}
Kwaliteit is een thema dat doorheen het volledige project wordt toegepast. Dit komt aan de hand van het project product description, deze wordt opgemaakt bij het starten van het project en gebruikt gedurende de rest van het project.

Product verwachtingen tonen aan wat er verwacht wordt van het product en waar acceptatie criteria aan hangen. De acceptatie criteria zijn meetbare waarden.

\subparagraph{Risico}
Risico's zijn onzekerheden die als ze voorkomen een effect hebben op de behaalbaarheid van doelstellingen. Het is heel belangrijk om alle risico's die het behalen van een project kunnen verhinderen op te stellen.

Als risico's staan voor onzekerheden, staan problemen voor zekerheden die gebeurd zijn of gaan gebeuren die niet gepland waren. Deze moeten bekeken worden door de project manager.

\paragraph{Processen}

\subparagraph{Starting up a project}
De corporate- en program management zal een mandaat opstellen waardoor men over het idee gaat beginnen nadenken. 

Een project begint altijd met een executive- en project manager toe te wijzen. Er wordt dan een klein tussenstapje genomen voor te reflecteren op vorige ervaringen, om dezelfde fouten niet nogmaals te begaan. 

Dan maak je de outline business case, voor een goeie rechtvaardiging te hebben om het project te starten. Tijdens deze stap wordt er ook gezocht wie er nog nodig is voor het project tot een succesvol einde te brengen, in de vorm van een project management team. Zo zorg je ervoor dat alle stakeholders hun stem kunnen uiten op het project. 

Hierna wordt de aanpak van het project besproken, en alle informatie samengebracht in een project-brief.
De laatste stap is de initiation stage.

\subparagraph{Directing a project}
De project board krijgt de project-brief en het initiation stage plan. Zij zullen dan toestemming geven om naar het volgende process te gaan.

Ook wordt dit process, directing a project aangesproken elke keer als men aan een nieuwe stage wilt beginnen, doordat hiervoor toestemming van het project board nodig is om te beginnen.

\subparagraph{Initiating a project}
Dit is waar de echte plannen van het project worden opgesteld.

Risk management strategy, Quality management strategy en Configuration management strategy worden voorbereid. Daarna wordt Communication management strategy ingepland. 
[Uitleggen]----------

Het project plan wordt opgesteld en de project controls worden besproken. Hierin wordt onder andere besproken hoe vaak er overlegd gaat worden met de project board.
[project controls uitleggen]----------------

Al deze informatie wordt gebruikt om de business case aan te vullen en wordt samengestoken in de Project Initiation Documentation. Hier zal de rechtvaardiging van het project inzitten alsook de tijdschemas en kostprijzen.

\subparagraph{Controlling a stage}
Wanneer het project board toestemming heeft gegeven gaat men naar de volgende fase. Dit is de plaats waar de project manager de meeste tijd zal spenderen tijdens een project. Deze kan dan een workpackage doorgeven aan de verschillende teams.

\subparagraph{Managing product delivery}
Een workpackage wordt gegeven aan de teams. Het workpackage is alles wat de team manager nodig heeft om te weten wat er moet gedaan worden. Een workpackage staat dus op het niveau van de team manager. 

Wanneer de team manager zo’n workpackage ontvangt, kan men er een team plan van opstellen zodat men weet hoe men de taken uit het workpackage moet uitvoeren. Tijdens het verwerken van het team plan moeten de teams regelmatig de project manager op de hoogte houden van de vooruitgang. Dit is het punt waar ze checkpoint reports gaan sturen naar de project manager. Zo kan de project manager dan de vooruitgang van het project bekijken. 

De project manager zal dan op zijn beurt highlight reports moeten sturen naar het project board. Project board moet dan op hun beurt het corporate or programme management up-to-date houden door highlight reports. Hier wordt natuurlijk ook gebruik gemaakt van het principe beheer in uitzondering.

\subparagraph{Managing a stage boundary}
Tegen het einde van een fase kom je in managing a stage boundary. Hier zal het plannen van de volgende fase plaatsvinden, het project plan updaten en een report maken over de afgelopen fase. 

Dit wordt dan gegeven aan de project board voor toestemming te krijgen om naar de volgende fase te gaan en deze lus zal herhaald worden totdat het project ten einde is. 

Aan het einde van de laatste fase zal dit process niet van start gaan, maar zal Closing a project beginnen.
\subparagraph{Closing a project}
Als alle fases doorlopen zijn, komt men op het punt om het project te sluiten. Men moet er zeker van zijn dat alle producten compleet zijn. 

Men gaat hier ook het project evalueren om eventuele lessen te trekken uit de vooruitgang, bestuur, enzovoort … en te hanteren bij volgende projecten. Dit wordt dan naar het project board gestuurd waar zij het project officieel gaan beëindigen.

\subsubsection{PMBOK}

\subsection{Hypothese}
Zowel Prince2 als PMBOK zijn combineerbaar met Agile.

\section{Vragen voor het interview}
\begin{itemize}
	\item Werken jullie met prince2 of pmbok
	\begin{itemize}
		\item waarom hebben jullie geopteerd om deze project management strategie te nemen.
		\item Tot nu toe, hoe hebben jullie deze strategie ervaren?
	\end{itemize}
		
	\item Hebben jullie ervaring met Agile werken?
	\begin{itemize}
		\item Waarom hebben jullie gekozen om niet Agile te werken?
	\end{itemize}
		
	\item Op welke vlakken hebben jullie je moeten aanpassen aan het Agile werken?
	\begin{itemize}
		\item Welke moeilijkheden hebben jullie moeten trotseren?
	\end{itemize}
		
	\item Welke voordelen hebben jullie ervaren aan Agile?
	\begin{itemize}
		\item Wat zijn mogelijke nadelen dat ook zijn opgedoken?
	\end{itemize}
		
	\item -Hoe zien jullie de toekomst van Project Management in prince2/PMBOK?
	
	\item PRINCE2
	\begin{itemize}
		\item Vanaf welk punt in een Prince2 project vind je dat je Agile aan het werken bent?
		\item Op welke mannier verschilt Prince2 van Agile?
		\item Wat zijn volgens jou de gelijkenissen?
		\item Als je iets zou kunnen veranderen/verbeteren in de werkwijze van Prince2 en Agile, wat zou dit zijn?
	\end{itemize}
		
	\item PMBOK
	\begin{itemize}
		\item In welke proces groups wordt er Agile gewerkt?
		\item Op welke mannier verschilt PMBOK van Agile?
		\item Wat zijn volgens jou de gelijkenissen?
		\item Als je iets zou kunnen veranderen/verbeteren in de werkwijze van PMBOK en Agile, wat zou dit zijn?
	\end{itemize}
		
\end{itemize}

\section{Analyse van het interview}

\subsection{Antwoorden}
test
\subsection{Analyse antwoorden}
test

\section{Reflectie}

\section{Bronvermelding}
\printbibliography
\end{document}
