\documentclass[]{article}
\usepackage[utf8]{inputenc}
\usepackage[backend=biber]{biblatex}
\addbibresource{case.bib}

%opening
\title{Hoe combineer je PRINCE2 of PMBOK met agile werken?}
\author{Van Loo Lieven, Van Looy Robbe, Storme Flor, Zwartjes Tijl}

\begin{document}

\maketitle
\begin{abstract}

\end{abstract}

\tableofcontents


\section{LiteratuurStudie}
\subsection{Samenvatting}
test
\subsection{Hypothese}

\section{Vragen voor het interview}
-Werken jullie met prince2 of pmbok
	waarom hebben jullie geopteerd om deze project management strategie te nemen.
	tot nu toe, hoe hebben jullie deze strategie ervaren?
-Hebben jullie ervaring met Agile werken?
	N- Waarom hebben jullie gekozen om niet Agile te werken?
-Op welke vlakken hebben jullie je moeten aanpassen aan het Agile werken?
	Welke moeilijkheden hebben jullie moeten trotseren?
-Welke voordelen hebben jullie ervaren aan Agile?
	Wat zijn mogelijke nadelen dat ook zijn opgedoken?
-Hoe zien jullie de toekomst van Project Management in prince2/PMBOK?
----
PRINCE2
	Vanaf welk punt in een Prince2 project vind je dat je Agile aan het werken bent?
	Op welke mannier verschilt Prince2 van Agile?
		Wat zijn volgens jou de gelijkenissen?
	Als je iets zou kunnen veranderen/verbeteren in de werkwijze van Prince2 en Agile, wat zou dit zijn?
	
----
PMBOK
	In welke proces groups wordt er Agile gewerkt?
	Op welke mannier verschilt PMBOK van Agile?
	Wat zijn volgens jou de gelijkenissen?
	Als je iets zou kunnen veranderen/verbeteren in de werkwijze van PMBOK en Agile, wat zou dit zijn?

\section{Analyse van het interview}

\subsection{Antwoorden}
test
\subsection{Analyse antwoorden}
test

\section{Reflectie}

\section{Bronvermelding}
\printbibliography
\end{document}
