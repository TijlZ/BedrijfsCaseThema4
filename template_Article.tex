\documentclass[]{article}
\usepackage[utf8]{inputenc}
\usepackage[backend=biber]{biblatex}
\usepackage{titlesec}
\addbibresource{case.bib}

\titleformat{\paragraph}
{\normalfont\normalsize\bfseries}{\theparagraph}{1em}{}
\titlespacing*{\paragraph}
{0pt}{3.25ex plus 1ex minus .2ex}{1.5ex plus .2ex}

\titleformat{\subparagraph}
{\normalfont\footnotesize\bfseries}{\thesubparagraph}{1em}{}
\titlespacing*{\subparagraph}
{0pt}{3.25ex plus 1ex minus .2ex}{1.5ex plus .2ex}

%opening
\title{Hoe combineer je PRINCE2 of PMBOK met agile werken?}
\author{Thema 4: Van Loo Lieven, Van Looy Robbe, Storme Flor, Zwartjes Tijl}

\begin{document}

\maketitle
\begin{abstract}

\end{abstract}

\tableofcontents


\section{LiteratuurStudie}
\subsection{Samenvatting}

\subsubsection{PRINCE2}
PRINCE2 is een methode voor projectmanagement. De methode is ontwikkeld door de Britse semioverheidsorganisatie Office Of Government Commerce. De PRINCE2-methode omvat 4 elementen: principes, thema’s, processen en de projectomgeving. Er zijn 7 principes en deze bestaan uit verplichtingen en best practices. Alle principes moeten worden gevolgd, anders wordt het project niet beschouwt als een PRINCE2 project.  Dit zijn dus de onderliggende regels die er zijn die je automatisch doet. Als deze principes niet juist worden toegepast, voelt dit raar. De thema’s worden voortdurend behandeld doorheen het project. Deze hebben we nodig doorheen het project, maar kunnen tijdens het project veranderen. De processen zijn de stapsgewijze vooruitgang doorheen een project. Ze vertellen wat er moet gebeuren met de thema’s op een bepaald punt in het project, niet vergeten, gebruikmakend van de principes. Bij het laatste element kijkt men naar de omgeving, de grootte, de complexiteit van het project en er zeker van zijn dat het goed bij het project past.

\paragraph{Principes}

\subparagraph{Zakelijke rechtvaardiging}
Er moet een gerechtvaardigde reden zijn om het project door te zetten. Als er geen reden is, dan moet het project afgesloten worden.

\subparagraph{Van ervaring leren}
Om een project in goede banen te leiden, moet er zich gebaseerd worden op ervaringen van vorige projecten. 

\subparagraph{Gedefinieerde rollen en verantwoordelijkheden}
Het projectteam moet een duidelijke organisatiestructuur bevatten. Iedereen moet z’n rol en de corresponderende verantwoordelijkheid kennen.

\subparagraph{Gedefinieerde rollen en verantwoordelijkheden}

\subparagraph{Beheer in fasen}
Complexe taken moeten worden opgesplitst in afzonderlijke fasen, dit worden management fasen genoemd. Een PRINCE2 project wordt fase per fase uitgedacht, gemonitord en gecontroleerd. De fasen worden gescheiden door beslissingspunten door het projectmanagement. Aan het einde van elke fase wordt er beslist op basis van de resultaten van de voorgaande fasen of men al dan niet mag beginnen aan de volgende fase. Dit is de keuze van het projectboard, dit heeft een grotere controle over het project en dus meer controle voor het senior management. Enkele voordelen van het faseren van een project:
\begin{itemize}
\item Het project verdelen in minder complexe delen.
\item Een gedetailleerd fase plan.
\item Zorgt voor leerrijke omgevingen voor volgende fasen.
\end{itemize}

\subparagraph{Beheer in uitzondering}
Dit wordt gebruikt om de ondergeschikte laag te besturen. Als er een groot probleem zich voordoet, een ‘Exception’ of ‘Uitzondering’, dan wilt dit zeggen dat het probleem boven de vastgelegde tolerantiegrens gaat. Een lid van het projectboard zal nooit iets horen van de projectmanager als het probleem de tolerantiegrens niet overschrijd. PRINCE2 definieert 6 toleranties. [NOG ERBIJ ZETTEN]

\subparagraph{Zich richten op de producten}
Een goede productomschrijving is zeer belangrijk tijdens het leiden van het project. Elke stakeholder moet hetzelfde product in gedachten hebben. De productomschrijving moet zo snel en duidelijk mogelijk worden geconstrueerd.

\subparagraph{Zich richten op de producten}
Een goede productomschrijving is zeer belangrijk tijdens het leiden van het project. Elke stakeholder moet hetzelfde product in gedachten hebben. De productomschrijving moet zo snel en duidelijk mogelijk worden geconstrueerd.

\subparagraph{Op maat van de projectomgeving}
Een PRINCE2 project moet op maat gemaakt zijn van het project op vlak van complexiteit, risico, omgeving, …. PRINCE2 is gemaakt voor alle types van projecten, maar maakt snel gemakkelijke projecten moeilijk. Daarom is het belangrijk om PRINCE2 te versimpelen aan het project. [MOGELIJKHEID TOT UITBEREIDING]


\paragraph{Thema's}
\subparagraph{Business Case}
De Business Case biedt een structuur om te beoordelen of de Business Case wenselijk, levensvatbaar en haalbaar is en de investering die tijdens het project wordt gedaan, waard is. [NOG VERDER UITWERKEN]

\subparagraph{Wijziging}

\subparagraph{Organisatie}
Dit staat voor alle mensen dat we nodig hebben op een project. Belangrijkste van al is de Corporate- of programma management. Zij staan er voor om het project te initialiseren, zonder hen wordt zelf niet gekeken om te beginnen aan het project.

In het project management team heb je: Project board die het project stuurt, Project manager die het project beheert in naam van de project board en de Team manager die staat voor het inleveren van de feature products. Dit zijn de 3 levels van management.

\subparagraph{Plannen}

\subparagraph{Voortgang}

\subparagraph{Kwaliteit}

\subparagraph{Risico}
[BEN IK MEE BEZIG -Tijl]

\paragraph{Processen}

\subparagraph{Starting up a project}
Een mandaat is wat er doet nadenken over een project. Dit doet een project echt starten. 

Een project begint altijd met een executive- en project manager toe te wijzen. Er wordt dan een klein tussenstapje genomen voor te reflecteren op verleden ervaringen, voor dezelfde fouten niet nogmaals te begaan. 

Dan maak je de omliggende business case, voor een goeie 'waarom' te hebben dat je dit project doet. Tijdens deze stap wordt er ook gezocht wie er nog nodig is voor het project tot een succesvol einde te brengen, in de vorm van een project management team. Zo zorg je ervoor dat alle stakeholders hun stem kunnen uiten op het project. 

Hierna wordt de aanpak van het project besproken, en alle informatie samengebracht in een project Brief.
De laatste stap is de initiation stage inplannen.

\subparagraph{Directing a project}
De project board krijgt de project-brief en de initiation stage plan. Zij zullen dan toestemming geven om naar de Initiating a project process te gaan. 

[AAN TE VULLEN]


\subparagraph{Initiating a project}
Dit is waar het echt plannen van het project plaatsvindt.

Risk management strategy, Quality management strategy en Configuration management strategy worden voorbereid.Daarna wordt de Communication management strategy ingepland. 

Het project plan wordt opgesteld en de project controls worden besproken. Hierin wordt onder andere besproken wat de frequentie is wanneer er bespreking is met de project board.

Al deze informatie wordt gebruikt om de business case aan te vullen en wordt samengestoken in de Project Initiation Documentation. Hier zal nu al business justification inzitten door de tijdschemas en costprijzen die inbegrepen zijn.

\subparagraph{Managing a stage boundary}

\subparagraph{Controlling a stage}

\subparagraph{Managing product delivery}

\subparagraph{Closing a project}


\subsection{Hypothese}
Zowel Prince2 als PMBOK zijn combineerbaar met Agile.

\section{Vragen voor het interview}
\begin{itemize}
	\item Werken jullie met prince2 of pmbok
	\begin{itemize}
		\item waarom hebben jullie geopteerd om deze project management strategie te nemen.
		\item Tot nu toe, hoe hebben jullie deze strategie ervaren?
	\end{itemize}
		
	\item Hebben jullie ervaring met Agile werken?
	\begin{itemize}
		\item Waarom hebben jullie gekozen om niet Agile te werken?
	\end{itemize}
		
	\item Op welke vlakken hebben jullie je moeten aanpassen aan het Agile werken?
	\begin{itemize}
		\item Welke moeilijkheden hebben jullie moeten trotseren?
	\end{itemize}
		
	\item Welke voordelen hebben jullie ervaren aan Agile?
	\begin{itemize}
		\item Wat zijn mogelijke nadelen dat ook zijn opgedoken?
	\end{itemize}
		
	\item -Hoe zien jullie de toekomst van Project Management in prince2/PMBOK?
	
	\item PRINCE2
	\begin{itemize}
		\item Vanaf welk punt in een Prince2 project vind je dat je Agile aan het werken bent?
		\item Op welke mannier verschilt Prince2 van Agile?
		\item Wat zijn volgens jou de gelijkenissen?
		\item Als je iets zou kunnen veranderen/verbeteren in de werkwijze van Prince2 en Agile, wat zou dit zijn?
	\end{itemize}
		
	\item PMBOK
	\begin{itemize}
		\item In welke proces groups wordt er Agile gewerkt?
		\item Op welke mannier verschilt PMBOK van Agile?
		\item Wat zijn volgens jou de gelijkenissen?
		\item Als je iets zou kunnen veranderen/verbeteren in de werkwijze van PMBOK en Agile, wat zou dit zijn?
	\end{itemize}
		
\end{itemize}

\section{Analyse van het interview}

\subsection{Antwoorden}
test
\subsection{Analyse antwoorden}
test

\section{Reflectie}

\section{Bronvermelding}
\printbibliography
\end{document}
