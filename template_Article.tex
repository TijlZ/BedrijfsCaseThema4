\documentclass[]{article}
\usepackage[utf8]{inputenc}
\usepackage[backend=biber]{biblatex}
\usepackage{titlesec}
\addbibresource{case.bib}

\titleformat{\paragraph}
{\normalfont\normalsize\bfseries}{\theparagraph}{1em}{}
\titlespacing*{\paragraph}
{0pt}{3.25ex plus 1ex minus .2ex}{1.5ex plus .2ex}

\titleformat{\subparagraph}
{\normalfont\footnotesize\bfseries}{\thesubparagraph}{1em}{}
\titlespacing*{\subparagraph}
{0pt}{3.25ex plus 1ex minus .2ex}{1.5ex plus .2ex}

%opening
\title{Hoe combineer je PRINCE2 of PMBOK met agile werken?}
\author{Thema 4: Van Loo Lieven, Van Looy Robbe, Storme Flor, Zwartjes Tijl}

\begin{document}

\maketitle
\begin{abstract}

\end{abstract}

\tableofcontents


\section{LiteratuurStudie}
\subsection{Samenvatting}

\subsubsection{PRINCE2}
PRINCE2 is een methode voor projectmanagement. De methode is ontwikkeld door de Britse semioverheidsorganisatie Office Of Government Commerce. De PRINCE2-methode omvat 4 elementen: principes, thema’s, processen en de projectomgeving. Er zijn 7 principes en deze bestaan uit verplichtingen en best practices. Alle principes moeten worden gevolgd, anders wordt het project niet beschouwt als een PRINCE2 project.  Dit zijn dus de onderliggende regels die er zijn die je automatisch doet. Als deze principes niet juist worden toegepast, voelt dit raar. De thema’s worden voortdurend behandeld doorheen het project. Deze hebben we nodig doorheen het project, maar kunnen tijdens het project veranderen. De processen zijn de stapsgewijze vooruitgang doorheen een project. Ze vertellen wat er moet gebeuren met de thema’s op een bepaald punt in het project, niet vergeten, gebruikmakend van de principes. Bij het laatste element kijkt men naar de omgeving, de grootte, de complexiteit van het project en er zeker van zijn dat het goed bij het project past.

\paragraph{Principes}

\subparagraph{Zakelijke rechtvaardiging}
Er moet een gerechtvaardigde reden zijn om het project door te zetten. Als er geen reden is, dan moet het project afgesloten worden.

\subparagraph{Van ervaring leren}
Om een project in goede banen te leiden, moet er zich gebaseerd worden op ervaringen van vorige projecten. 

\subparagraph{Gedefinieerde rollen en verantwoordelijkheden}
Het projectteam moet een duidelijke organisatiestructuur bevatten. Iedereen moet z’n rol en de corresponderende verantwoordelijkheid kennen.

\subparagraph{Gedefinieerde rollen en verantwoordelijkheden}

\subparagraph{Beheer in fasen}
Complexe taken moeten worden opgesplitst in afzonderlijke fasen, dit worden management fasen genoemd. Een PRINCE2 project wordt fase per fase uitgedacht, gemonitord en gecontroleerd. De fasen worden gescheiden door beslissingspunten door het projectmanagement. Aan het einde van elke fase wordt er beslist op basis van de resultaten van de voorgaande fasen of men al dan niet mag beginnen aan de volgende fase. Dit is de keuze van het projectboard, dit heeft een grotere controle over het project en dus meer controle voor het senior management. Enkele voordelen van het faseren van een project:
\begin{itemize}
\item Het project verdelen in minder complexe delen.
\item Een gedetailleerd fase plan.
\item Zorgt voor leerrijke omgevingen voor volgende fasen.
\end{itemize}

\subparagraph{Beheer in uitzondering}
Dit wordt gebruikt om de ondergeschikte laag te besturen. Als er een groot probleem zich voordoet, een ‘Exception’ of ‘Uitzondering’, dan wilt dit zeggen dat het probleem boven de vastgelegde tolerantiegrens gaat. Een lid van het projectboard zal nooit iets horen van de projectmanager als het probleem de tolerantiegrens niet overschrijd. PRINCE2 definieert 6 toleranties. [NOG ERBIJ ZETTEN]

\subparagraph{Zich richten op de producten}
Een goede productomschrijving is zeer belangrijk tijdens het leiden van het project. Elke stakeholder moet hetzelfde product in gedachten hebben. De productomschrijving moet zo snel en duidelijk mogelijk worden geconstrueerd.

\subparagraph{Zich richten op de producten}
Een goede productomschrijving is zeer belangrijk tijdens het leiden van het project. Elke stakeholder moet hetzelfde product in gedachten hebben. De productomschrijving moet zo snel en duidelijk mogelijk worden geconstrueerd.

\subparagraph{Op maat van de projectomgeving}
Een PRINCE2 project moet op maat gemaakt zijn van het project op vlak van complexiteit, risico, omgeving, …. PRINCE2 is gemaakt voor alle types van projecten, maar maakt snel gemakkelijke projecten moeilijk. Daarom is het belangrijk om PRINCE2 te versimpelen aan het project. [MOGELIJKHEID TOT UITBEREIDING]


\paragraph{Thema's}
\subparagraph{Business Case}
De Business Case biedt een structuur om te beoordelen of de Business Case wenselijk, levensvatbaar en haalbaar is en de investering die tijdens het project wordt gedaan, waard is. [NOG VERDER UITWERKEN]

\subparagraph{Wijziging}

\subparagraph{Organisatie}
Dit staat voor alle mensen dat we nodig hebben op een project. Belangrijkste van al is de Corporate- of programma management. Zij staan er voor om het project te initialiseren, zonder hen wordt zelf niet gekeken om te beginnen aan het project.

In het project management team heb je: Project board die het project stuurt, Project manager die het project beheert in naam van de project board en de Team manager die staat voor het inleveren van de feature products. Dit zijn de 3 levels van management.

\subparagraph{Plannen}

\subparagraph{Voortgang}

\subparagraph{Kwaliteit}

\subparagraph{Risico}
[BEN IK MEE BEZIG -Tijl]

\paragraph{Processen}

\subparagraph{Starting up a project}
Een mandaat is wat er doet nadenken over een project. Dit doet een project echt starten. 

Een project begint altijd met een executive- en project manager toe te wijzen. Er wordt dan een klein tussenstapje genomen voor te reflecteren op verleden ervaringen, voor dezelfde fouten niet nogmaals te begaan. 

Dan maak je de omliggende business case, voor een goeie 'waarom' te hebben dat je dit project doet. Tijdens deze stap wordt er ook gezocht wie er nog nodig is voor het project tot een succesvol einde te brengen, in de vorm van een project management team. Zo zorg je ervoor dat alle stakeholders hun stem kunnen uiten op het project. 

Hierna wordt de aanpak van het project besproken, en alle informatie samengebracht in een project Brief.
De laatste stap is de initiation stage inplannen.

\subparagraph{Directing a project}
De project board krijgt de project-brief en de initiation stage plan. Zij zullen dan toestemming geven om naar de Initiating a project process te gaan. 

[AAN TE VULLEN]


\subparagraph{Initiating a project}
Dit is waar het echt plannen van het project plaatsvindt.

Risk management strategy, Quality management strategy en Configuration management strategy worden voorbereid.Daarna wordt de Communication management strategy ingepland. 

Het project plan wordt opgesteld en de project controls worden besproken. Hierin wordt onder andere besproken wat de frequentie is wanneer er bespreking is met de project board.

Al deze informatie wordt gebruikt om de business case aan te vullen en wordt samengestoken in de Project Initiation Documentation. Hier zal nu al business justification inzitten door de tijdschemas en costprijzen die inbegrepen zijn.

\subparagraph{Managing a stage boundary}

\subparagraph{Controlling a stage}

\subparagraph{Managing product delivery}

\subparagraph{Closing a project}

\subsubsection{PMBoK}
PMBok staat voluit voor Project Management Body of Knowledge. Het is geen methode voor project management (zoals PRINCE2), maar eerder een verzameling van best practices. Het framework wordt onderhouden door het PMI (Project Management Institute) die er een handleiding voor uitbrachten genaamd: "A Guide to the Project Management Body of Knowledge (PMBOK\textsuperscript{\textregistered} Guide)". Dit bevat een overzicht van de kennis die is gerangschikt volgens de verschillende aandachtsgebieden in projectmanagent. PMBoK wordt vooral gebruikt in Amerika en Azië. Dit in tegenstelling tot PRINCE2 dat vooral invloed heeft in Noord Europa en Groot Brittannië. Desondanks heeft PMBoK in de rest van de wereld ook veel invloed. Het PMI is ook continu bezig met het updaten van de informatie die hun guide bevat aangezien ook projectmanagement zich continu ontwikkeld. De PMBOK\textsuperscript{\textregistered} structuur, zoals die vandag beschreven staat, bestaat uit: 5 process groups, 10 knowledge areas en 47 project management processes.	
\paragraph{5 Process Groups}
\subparagraph{Integration}
De eerste process group, initiating, bevat de processen die nodig zijn om het begin van een project effectief te structureren. Hier wordt de visie bepaald van wat een succesvol project moet bereiken.
Het project wordt formeel geautoriseerd door de sponsor, de initiële scope wordt gedefinieerd, en de stakeholders worden geïdentificeerd. Ook wordt er een project manager vroeg toegewezen.
Deze process group is belangrijk omdat het project hier gealigneerd wordt met de strategische doelen van de organisatie. 
\subparagraph{Planning}
De planning process group voorziet de processen die: de scope van het project verder uitwerken, de flow maximaliseren door de plannen te voorzien en de prioriteiten en noden van het team samenstellen.
Bij de scope worden de risico’s, milestones en het budget gedefinieerd. Er wordt een iteratief planning proces genaamd progressive elaboration uitgevoerd, waarin gedetailleerde project documenten worden uitgewerkt. 
Deze process group bevat maar liefst 24 processen. 
\subparagraph{Executing}
In de executing process group worden de teams op een effectieve manier gestuurd. Er worden activiteiten gearrangeerd die overeenkomen met de timeline en de verwachtingen van de sponsor en stakeholders. 
Project managers zullen met deze skills een hoog niveau van organisatie en communicatie vertonen om de belangen van het team te adresseren en te zorgen dat de tijds-, budget- en kwaliteitsverwachtingen voldaan worden. 
In deze fase wordt het meeste van het budget gespendeerd en de te leveren producten geproduceerd. 
\subparagraph{Monitoring and Controlling}
Anders dan de andere process groups, gebeurt monitoring en controlling niet sequentieel, het wordt uitgevoerd over de hele levensduur van het project. 
Deze processen zijn nodig om de progressie van het project op te volgen, te reviewen en te reguleren.
Hier worden change orders vanuit de stakeholders geadresseerd, en worden gebieden van het project geïdentificeerd waar verandering nodig is en het initiële plan wordt aangepast waar nodig.
\subparagraph{Closing}
Bij de closing process group wordt het project definitief afgesloten, hier wordt er acceptatie verkregen van de klant, de documenten gearchiveerd, de geleerde lessen overlopen, de contracten beëindigd en wordt het team bedankt. 
\paragraph{10 Knowledge Areas}
\subparagraph{Integration}
Integratie draait om alle aparte deeltjes bij elkaar te brengen om zo het hele project te behandelen en niet de aparte processen.
\subparagraph{Scope}
De scope opstellen is één van de belangrijkste taken van de project manager. Dit maakt duidelijk wat er verwacht wordt van het project en wat buiten het project valt. Hier worden de requirements van het project besproken en de work breakdown structure opgesteld.
\subparagraph{Time}
In deze knowledge area, wordt het volledige timeframe van het project vastgelegd. Er wordt beslist hoeveel tijd er zal besteed worden aan ieder stukje, de volgorde van die stukken en wanneer het volledige project af moet zijn.
\subparagraph{Cost}
Hier wordt er besproken hoeveel elk onderdeel van het project zal kosten. Aan de hand daarvan kan er dan gekeken worden hoeveel het totale project zal kosten. Dit is een belangrijk onderdeel, want er moet ook gecontroleerd worden of het project weldegelijk past binnen de afgesproken budgetten.
\subparagraph{Quality}
In deze area wordt er kwaliteitscontrole en kwaliteitsmanagement toegevoegd aan het project. Deze worden opgesteld op basis van de eisen van de opdrachtgever. Dit zorgt ervoor dat het project voldoet aan de verwachtingen van de opdrachtgever.
\subparagraph{Procurement}
Tijdens de procurement moet je alle middelen voorzien die er nodig zijn om het project te kunnen uitvoeren en manage. De project manager moet er op toezien op een manier dat de middelen tijdig, in de juiste kwaliteit en hoeveelheid beschikbaar zijn.
\subparagraph{Human resources}
Nadat alle middelen zijn verzameld om te beginnen aan het project, moet er een team worden samengesteld. Nu is het de taak van de project manager om een team samen te stellen dat de nodige competenties al bezitten of deze kunnen vergaren. Ze volgen hen ook mee op om team geëngageerd te houden doorheen het project.
\subparagraph{Communications}
Er moet een duidelijk plan worden opgesteld voor de communicatie. Communicatie is een heel belangrijk onderdeel van een project en er moet dus voor gezorgd worden dat iedereen altijd op de hoogte is van de nieuwste informatie.
\subparagraph{Risk management}
Alle risico's die worden vastgesteld vooraf of tijdens het project moeten worden gemanaged. De risico's moeten worden geregistreerd, het effectieve risico moet worden ingeschat en dan moeten er maatregelen worden voorbereid/genomen.
\subparagraph{Stakeholder management}
De stakeholders zijn alle partijen die belang hebben bij het slagen van het project. Met hen moet er dus zeker ook rekening gehouden worden. Hun eisen en wensen moeten vooraf en tijdens het project dus ook in kaart gebracht worden en verwoven worden in het project.

\subsection{Hypothese}
Zowel Prince2 als PMBOK zijn combineerbaar met Agile.

\section{Vragen voor het interview}
\begin{itemize}
	\item Werken jullie met prince2 of pmbok
	\begin{itemize}
		\item waarom hebben jullie geopteerd om deze project management strategie te nemen.
		\item Tot nu toe, hoe hebben jullie deze strategie ervaren?
	\end{itemize}
		
	\item Hebben jullie ervaring met Agile werken?
	\begin{itemize}
		\item Waarom hebben jullie gekozen om niet Agile te werken?
	\end{itemize}
		
	\item Op welke vlakken hebben jullie je moeten aanpassen aan het Agile werken?
	\begin{itemize}
		\item Welke moeilijkheden hebben jullie moeten trotseren?
	\end{itemize}
		
	\item Welke voordelen hebben jullie ervaren aan Agile?
	\begin{itemize}
		\item Wat zijn mogelijke nadelen dat ook zijn opgedoken?
	\end{itemize}
		
	\item -Hoe zien jullie de toekomst van Project Management in prince2/PMBOK?
	
	\item PRINCE2
	\begin{itemize}
		\item Vanaf welk punt in een Prince2 project vind je dat je Agile aan het werken bent?
		\item Op welke mannier verschilt Prince2 van Agile?
		\item Wat zijn volgens jou de gelijkenissen?
		\item Als je iets zou kunnen veranderen/verbeteren in de werkwijze van Prince2 en Agile, wat zou dit zijn?
	\end{itemize}
		
	\item PMBOK
	\begin{itemize}
		\item In welke proces groups wordt er Agile gewerkt?
		\item Op welke mannier verschilt PMBOK van Agile?
		\item Wat zijn volgens jou de gelijkenissen?
		\item Als je iets zou kunnen veranderen/verbeteren in de werkwijze van PMBOK en Agile, wat zou dit zijn?
	\end{itemize}
		
\end{itemize}

\section{Analyse van het interview}

\subsection{Antwoorden}
test
\subsection{Analyse antwoorden}
test

\section{Reflectie}

\section{Bronvermelding}
\printbibliography
\end{document}
