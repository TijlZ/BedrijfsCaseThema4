\documentclass[]{article}
\usepackage[utf8]{inputenc}
\usepackage[backend=biber]{biblatex}
\usepackage{titlesec}
\addbibresource{case.bib}

\titleformat{\paragraph}
{\normalfont\normalsize\bfseries}{\theparagraph}{1em}{}
\titlespacing*{\paragraph}
{0pt}{3.25ex plus 1ex minus .2ex}{1.5ex plus .2ex}

\titleformat{\subparagraph}
{\normalfont\footnotesize\bfseries}{\thesubparagraph}{1em}{}
\titlespacing*{\subparagraph}
{0pt}{3.25ex plus 1ex minus .2ex}{1.5ex plus .2ex}

%opening
\title{Hoe combineer je PRINCE2 of PMBOK met agile werken?}
\author{Thema 4: Van Loo Lieven, Van Looy Robbe, Storme Flor, Zwartjes Tijl}

\begin{document}

\maketitle
\begin{abstract}

\end{abstract}

\tableofcontents


\section{LiteratuurStudie}
\subsection{Samenvatting}

\subsubsection{PRINCE2}
PRINCE2 is een methode voor projectmanagement. De methode is ontwikkeld door de Britse semioverheidsorganisatie Office Of Government Commerce. De PRINCE2-methode omvat 4 elementen: principes, thema’s, processen en de projectomgeving. Er zijn 7 principes en deze bestaan uit verplichtingen en best practices. Alle principes moeten worden gevolgd, anders wordt het project niet beschouwt als een PRINCE2 project.  Dit zijn dus de onderliggende regels die er zijn die je automatisch doet. Als deze principes niet juist worden toegepast, voelt dit raar. De thema’s worden voortdurend behandeld doorheen het project. Deze hebben we nodig doorheen het project, maar kunnen tijdens het project veranderen. De processen zijn de stapsgewijze vooruitgang doorheen een project. Ze vertellen wat er moet gebeuren met de thema’s op een bepaald punt in het project, niet vergeten, gebruikmakend van de principes. Bij het laatste element kijkt men naar de omgeving, de grootte, de complexiteit van het project en er zeker van zijn dat het goed bij het project past.

\paragraph{Principes}

\subparagraph{Zakelijke rechtvaardiging}
Er moet een gerechtvaardigde reden zijn om het project door te zetten. Als er geen reden is, dan moet het project afgesloten worden.

\subparagraph{Van ervaring leren}
Om een project in goede banen te leiden, moet er zich gebaseerd worden op ervaringen van vorige projecten. 

\subparagraph{Gedefinieerde rollen en verantwoordelijkheden}
Het projectteam moet een duidelijke organisatiestructuur bevatten. Iedereen moet z’n rol en de corresponderende verantwoordelijkheid kennen.

\subparagraph{Gedefinieerde rollen en verantwoordelijkheden}
Een PRINCE2 project heeft vaste rollen, deze zijn:
\begin{itemize}
	\item Project Board \\ Deze zijn de 3 primaire stakeholders.
	\item Project Assurance \\ Zij verzekeren de belangen van de stakeholders.
	\item Change Authority \\ Maken beslissingen over change in naam van de project board.
	\item Project Manager \\ Staat in voor alledaagse management van het project.
	\item Project Support \\ Helpt de project manager.
	\item Team Manager \\ Garandeert de efficiëntie van de productie van het team.
	
\end{itemize}

\subparagraph{Beheer in fasen}
Complexe taken moeten worden opgesplitst in afzonderlijke fasen, dit worden management fasen genoemd. Een PRINCE2 project wordt fase per fase uitgedacht, gemonitord en gecontroleerd. De fasen worden gescheiden door beslissingspunten door het projectmanagement. Aan het einde van elke fase wordt er beslist op basis van de resultaten van de voorgaande fasen of men al dan niet mag beginnen aan de volgende fase. Dit is de keuze van het projectboard, dit heeft een grotere controle over het project en dus meer controle voor het senior management. Enkele voordelen van het faseren van een project:
\begin{itemize}
\item Het project verdelen in minder complexe delen.
\item Een gedetailleerd fase plan.
\item Zorgt voor leerrijke omgevingen voor volgende fasen.
\end{itemize}

\subparagraph{Beheer in uitzondering}
Dit wordt gebruikt om de ondergeschikte laag te besturen. Als er een groot probleem zich voordoet, een ‘Exception’ of ‘Uitzondering’, dan wilt dit zeggen dat het probleem boven de vastgelegde tolerantiegrens gaat. Een lid van het projectboard zal nooit iets horen van de projectmanager als het probleem de tolerantiegrens niet overschrijd.

\subparagraph{Zich richten op de producten}
Een goede productomschrijving is zeer belangrijk tijdens het leiden van het project. Elke stakeholder moet hetzelfde product in gedachten hebben. De productomschrijving moet zo snel en duidelijk mogelijk worden geconstrueerd.

\subparagraph{Zich richten op de producten}
Een goede productomschrijving is zeer belangrijk tijdens het leiden van het project. Elke stakeholder moet hetzelfde product in gedachten hebben. De productomschrijving moet zo snel en duidelijk mogelijk worden geconstrueerd.

\subparagraph{Op maat van de projectomgeving}
Een PRINCE2 project moet op maat gemaakt zijn van het project op vlak van complexiteit, risico, omgeving, …. PRINCE2 is gemaakt voor alle types van projecten, maar maakt snel gemakkelijke projecten moeilijk. Daarom is het belangrijk om te weten wanneer je PRINCE2 gebruikt of een versimpelde versie.


\paragraph{Thema's}
\subparagraph{Business Case}
De business case biedt een structuur om te beoordelen of het project wenselijk, levensvatbaar en haalbaar is en de investering die tijdens het project wordt gedaan, waard is. De business case is iets wat evolueert doorheen het project. 

Bij de start van het project wordt er een outline business case opgesteld, maar als we verdergaan door de initiatiefase wordt deze echter geüpdatet naar een volledig gedetailleerde business case. Na elke fase wordt er dan gekeken of er nog rechtvaardiging(zie principes) is om het project naar de volgende fase te laten gaan.

\subparagraph{Wijziging}
Wijziging is niet alleen kunnen omgaan met veranderingen in de planning die komen door beslissingen, maar ook het kunnen inschatten en voorbereiden op wijzigingen die moeten komen door problemen die opduiken. Dit gaat natuurlijk hand-in-hand met risk management \footnote{Bedrijfsrisico's in kaart brengen en beoordelen}.

Eén van de belangrijkste eigenschappen van een project manager is het kunnen weigeren of accepteren van voorgestelde wijzigingen. En als vaak vergeten deel, het bedanken van de persoon die het heeft voorgesteld.

Prince 2 heeft een vaste aanpak voor het documenteren van een wijziging, dit omvat 6 zaken:
\begin{itemize}
	\item Change Control Approach \\ Dat beschrijft hoe issues en wijzigingen aangepakt worden in het project.
	\item Configuration Item Record \\ Dit is een dataset voor elk product in het project.
	\item Product Status Account \\ Dit is een statusreport van elk product in het project.
	\item Daily Log \\ Hierin wordt door de project manager alle kleine informele zaken genoteerd.
	\item Issue Register \\ Dit is het formeel document voor de project manager.
	\item Issue Report \\ Hier wordt de issue in detail uitgelegd. Dit kan volgens PRINCE2 3 vormen hebben:Vraag voor wijziging, off-specification of probleem/vrees.
\end{itemize}

\subparagraph{Organisatie}
Dit staat voor alle mensen die men nodig heeft op een project. Belangrijkste van al is de Corporate- of program management. Zij staan er in om het project te initialiseren.

In het project management team heb je: een project board die het project stuurt, project manager die het project beheert in naam van het project board en de team manager die instaat voor het leveren van de feature products.

\subparagraph{Plannen}
Bij het plannen van een project wordt er meestal gebruik gemaakt van vorige projecten of soortgelijke projecten. Men kan aan de hand van die projecten het plan opstellen voor het huidig project. 

Eén van de eerste zaken die men moet proberen verkrijgen tijdens het plannen is een idee van de scope, een product breakdown structuur kan helpen en zorgt ervoor dat iedereen het plan snel begrijpt. 

\subparagraph{Voortgang}
Voortgang gaat vooral over hoe het project wordt gecontroleerd, waar de voortgang op het huidige moment zit en dit vergelijken met het plan dat werd opgesteld. Het voortgang thema kan uitgelegd worden aan de hand van drie delen. 

Het in het oog houden van de vooruitgang met de voorop geplande tijdsperiodes. Kunnen voorspellen, hoelang men ergens aan gaat bezig zijn en correct kunnen handelen als er een onverwacht obstakel zich voordoet.

\subparagraph{Kwaliteit}
Kwaliteit is een thema dat doorheen het volledige project wordt toegepast. Dit komt aan de hand van het project product description, deze wordt opgemaakt bij het starten van het project en gebruikt gedurende de rest van het project.

Product verwachtingen tonen aan wat er verwacht wordt van het product en waar acceptatie criteria aan hangen. De acceptatie criteria zijn meetbare waarden.

\subparagraph{Risico}
Risico's zijn onzekerheden die als ze voorkomen een effect hebben op de behaalbaarheid van doelstellingen. Het is heel belangrijk om alle risico's die het behalen van een project kunnen verhinderen op te stellen.

Als risico's staan voor onzekerheden, staan problemen voor zekerheden die gebeurd zijn of gaan gebeuren die niet gepland waren. Deze moeten bekeken worden door de project manager.

\paragraph{Processen}

\subparagraph{Starting up a project}
De corporate- en program management zal een mandaat opstellen waardoor men over het idee gaat beginnen nadenken. 

Een project begint altijd met een executive- en project manager toe te wijzen. Er wordt dan een klein tussenstapje genomen voor te reflecteren op vorige ervaringen, om dezelfde fouten niet nogmaals te begaan. 

Dan maak je de outline business case, voor een goeie rechtvaardiging te hebben om het project te starten. Tijdens deze stap wordt er ook gezocht wie er nog nodig is voor het project tot een succesvol einde te brengen, in de vorm van een project management team. Zo zorg je ervoor dat alle stakeholders hun stem kunnen uiten op het project. 

Hierna wordt de aanpak van het project besproken, en alle informatie samengebracht in een project-brief.
De laatste stap is de initiation stage.

\subparagraph{Directing a project}
De project board krijgt de project-brief en het initiation stage plan. Zij zullen dan toestemming geven om naar het volgende process te gaan.

Ook wordt dit process, directing a project aangesproken elke keer als men aan een nieuwe stage wilt beginnen, doordat hiervoor toestemming van het project board nodig is om te beginnen.

\subparagraph{Initiating a project}
Dit is waar de echte plannen van het project worden opgesteld.

Risk management strategy, Quality management strategy \footnote{gewenst niveau van kwaliteit te behalen en te behouden} en Configuration management strategy \footnote{registratie en actualisering van de informatie die de computersystemen van een onderneming beschrijft} worden voorbereid. Daarna wordt Communication management strategy \footnote{
	systematische planning, implementatie, monitoring en herziening van alle communicatiekanalen binnen een organisatie} ingepland.

Het project plan wordt opgesteld en de project controls \footnote{schaal, risico's en belang van project beheren} worden besproken. Hierin wordt onder andere besproken hoe vaak er overlegd gaat worden met de project board.

Al deze informatie wordt gebruikt om de business case aan te vullen en wordt samengestoken in de Project Initiation Documentation. Hier zal de rechtvaardiging van het project inzitten alsook de tijdschemas en kostprijzen.

\subparagraph{Controlling a stage}
Wanneer het project board toestemming heeft gegeven gaat men naar de volgende fase. Dit is de plaats waar de project manager de meeste tijd zal spenderen tijdens een project. Deze kan dan een workpackage doorgeven aan de verschillende teams.

\subparagraph{Managing product delivery}
Een workpackage wordt gegeven aan de teams. Het workpackage is alles wat de team manager nodig heeft om te weten wat er moet gedaan worden. Een workpackage staat dus op het niveau van de team manager. 

Wanneer de team manager zo’n workpackage ontvangt, kan men er een team plan van opstellen zodat men weet hoe men de taken uit het workpackage moet uitvoeren. Tijdens het verwerken van het team plan moeten de teams regelmatig de project manager op de hoogte houden van de vooruitgang. Dit is het punt waar ze checkpoint reports gaan sturen naar de project manager. Zo kan de project manager dan de vooruitgang van het project bekijken. 

De project manager zal dan op zijn beurt highlight reports moeten sturen naar het project board. Project board moet dan op hun beurt het corporate or programme management up-to-date houden door highlight reports. Hier wordt natuurlijk ook gebruik gemaakt van het principe beheer in uitzondering.

\subparagraph{Managing a stage boundary}
Tegen het einde van een fase kom je in managing a stage boundary. Hier zal het plannen van de volgende fase plaatsvinden, het project plan updaten en een report maken over de afgelopen fase. 

Dit wordt dan gegeven aan de project board voor toestemming te krijgen om naar de volgende fase te gaan en deze lus zal herhaald worden totdat het project ten einde is. 

Aan het einde van de laatste fase zal dit process niet van start gaan, maar zal Closing a project beginnen.
\subparagraph{Closing a project}
Als alle fases doorlopen zijn, komt men op het punt om het project te sluiten. Men moet er zeker van zijn dat alle producten compleet zijn. 

Men gaat hier ook het project evalueren om eventuele lessen te trekken uit de vooruitgang, bestuur, enzovoort … en te hanteren bij volgende projecten. Dit wordt dan naar het project board gestuurd waar zij het project officieel gaan beëindigen.

\subsubsection{PMBOK}
PMBok staat voluit voor Project Management Body of Knowledge. Het is geen methode voor project management (zoals PRINCE2), maar eerder een verzameling van best practices. Het framework wordt onderhouden door het PMI (Project Management Institute) die er een handleiding voor uitbrachten genaamd: "A Guide to the Project Management Body of Knowledge (PMBOK\textsuperscript{\textregistered} Guide)". Dit bevat een overzicht van de kennis die is gerangschikt volgens de verschillende aandachtsgebieden in projectmanagent. PMBoK wordt vooral gebruikt in Amerika en Azië. Dit in tegenstelling tot PRINCE2 dat vooral invloed heeft in Noord Europa en Groot Brittannië. Desondanks heeft PMBoK in de rest van de wereld ook veel invloed. Het PMI is ook continu bezig met het updaten van de informatie die hun guide bevat aangezien ook projectmanagement zich continu ontwikkeld. De PMBOK\textsuperscript{\textregistered} structuur, zoals die vandag beschreven staat, bestaat uit: 5 process groups, 10 knowledge areas en 47 project management processes.	
\paragraph{5 Process Groups}
\subparagraph{Initiating}
De eerste process group, initiating, bevat de processen die nodig zijn om het begin van een project effectief te structureren. Hier wordt de visie bepaald van wat een succesvol project moet bereiken.
Het project wordt formeel geautoriseerd door de sponsor, de initiële scope wordt gedefinieerd, en de stakeholders worden geïdentificeerd. Ook wordt er een project manager vroeg toegewezen.
Deze process group is belangrijk omdat het project hier gealigneerd wordt met de strategische doelen van de organisatie. 
\subparagraph{Planning}
De planning process group voorziet de processen die: de scope van het project verder uitwerken, de flow maximaliseren door de plannen te voorzien en de prioriteiten en noden van het team samenstellen.
Bij de scope worden de risico’s, milestones en het budget gedefinieerd. Er wordt een iteratief planning proces genaamd progressive elaboration uitgevoerd, waarin gedetailleerde project documenten worden uitgewerkt. 
Deze process group bevat maar liefst 24 processen. 
\subparagraph{Executing}
In de executing process group worden de teams op een effectieve manier gestuurd. Er worden activiteiten gearrangeerd die overeenkomen met de timeline en de verwachtingen van de sponsor en stakeholders. 
Project managers zullen met deze skills een hoog niveau van organisatie en communicatie vertonen om de belangen van het team te adresseren en te zorgen dat de tijds-, budget- en kwaliteitsverwachtingen voldaan worden. 
In deze fase wordt het meeste van het budget gespendeerd en de te leveren producten geproduceerd. 
\subparagraph{Monitoring and Controlling}
Anders dan de andere process groups, gebeurt monitoring en controlling niet sequentieel, het wordt uitgevoerd over de hele levensduur van het project. 
Deze processen zijn nodig om de progressie van het project op te volgen, te reviewen en te reguleren.
Hier worden change orders vanuit de stakeholders geadresseerd, en worden gebieden van het project geïdentificeerd waar verandering nodig is en het initiële plan wordt aangepast waar nodig.
\subparagraph{Closing}
Bij de closing process group wordt het project definitief afgesloten, hier wordt er acceptatie verkregen van de klant, de documenten gearchiveerd, de geleerde lessen overlopen, de contracten beëindigd en wordt het team bedankt. 
\paragraph{10 Knowledge Areas}
\subparagraph{Integration}
Integratie draait om alle aparte deeltjes bij elkaar te brengen om zo het hele project te behandelen en niet de aparte processen.
\subparagraph{Scope}
De scope opstellen is één van de belangrijkste taken van de project manager. Dit maakt duidelijk wat er verwacht wordt van het project en wat buiten het project valt. Hier worden de requirements van het project besproken en de work breakdown structure opgesteld.
\subparagraph{Time}
In deze knowledge area, wordt het volledige timeframe van het project vastgelegd. Er wordt beslist hoeveel tijd er zal besteed worden aan ieder stukje, de volgorde van die stukken en wanneer het volledige project af moet zijn.
\subparagraph{Cost}
Hier wordt er besproken hoeveel elk onderdeel van het project zal kosten. Aan de hand daarvan kan er dan gekeken worden hoeveel het totale project zal kosten. Dit is een belangrijk onderdeel, want er moet ook gecontroleerd worden of het project weldegelijk past binnen de afgesproken budgetten.
\subparagraph{Quality}
In deze area wordt er kwaliteitscontrole en kwaliteitsmanagement toegevoegd aan het project. Deze worden opgesteld op basis van de eisen van de opdrachtgever. Dit zorgt ervoor dat het project voldoet aan de verwachtingen van de opdrachtgever.
\subparagraph{Procurement}
Tijdens de procurement moet je alle middelen voorzien die er nodig zijn om het project te kunnen uitvoeren en manage. De project manager moet er op toezien op een manier dat de middelen tijdig, in de juiste kwaliteit en hoeveelheid beschikbaar zijn.
\subparagraph{Human resources}
Nadat alle middelen zijn verzameld om te beginnen aan het project, moet er een team worden samengesteld. Nu is het de taak van de project manager om een team samen te stellen dat de nodige competenties al bezitten of deze kunnen vergaren. Ze volgen hen ook mee op om team geëngageerd te houden doorheen het project.
\subparagraph{Communications}
Er moet een duidelijk plan worden opgesteld voor de communicatie. Communicatie is een heel belangrijk onderdeel van een project en er moet dus voor gezorgd worden dat iedereen altijd op de hoogte is van de nieuwste informatie.
\subparagraph{Risk management}
Alle risico's die worden vastgesteld vooraf of tijdens het project moeten worden gemanaged. De risico's moeten worden geregistreerd, het effectieve risico moet worden ingeschat en dan moeten er maatregelen worden voorbereid/genomen.
\subparagraph{Stakeholder management}
De stakeholders zijn alle partijen die belang hebben bij het slagen van het project. Met hen moet er dus zeker ook rekening gehouden worden. Hun eisen en wensen moeten vooraf en tijdens het project dus ook in kaart gebracht worden en verwoven worden in het project.

\subsection{Hypothese}
Zowel Prince2 als PMBOK zijn combineerbaar met Agile.

\section{Vragen voor het interview}
\begin{itemize}
	\item Heeft u met Prince2 of PMBOK gewerkt?
	\begin{itemize}
		\item Welk Prince2 certificaat heeft u behaald?
		\item Hoeveel jaar geleden was dit?
		\item Was dit al binnen de Chronos group?
		\item Welke rol had u binnen het project management team?
	\end{itemize}
	\item Waarom heeft u voor een Prince2 certificaat gekozen?
	\item Werkte u toen al Agile?
	\item Wat vond u van de Prince2 methodologie?
	\item Nu bestaat er een Prince2 Agile certificaat, wat vind u hiervan?
	\item Vanaf welk punt denkt u dat het best is om van Prince2 naar een meer Agile werkmethode over te schakelen?
	\begin{itemize}
		\item Wat zou volgens u het ideale tijdsschema voor de project brief te reviewen?
	\end{itemize}
	\item Welke items van Prince2 mist u binnen Agile?
	\item Welke moeilijkheden heeft u meegemaakt in de switch van Prince2 naar volledig Agile werken?
	\begin{itemize}
		\item Welke aanpassingen in denkwijze heeft u ondervonden?
	\end{itemize}
	\item Hoe denkt u dat Prince2 en Agile zouden kunnen samenwerken?
	\begin{itemize}
		\item Hoe zou u Prince2 en Agile combineren?
		\item Hoe zou men reageren 10 jaar geleden als er een verandering zou komen om Prince2 meer Agile te maken?
	\end{itemize}
\end{itemize}

\section{Analyse van het interview}

\subsection{Antwoorden}
\begin{itemize}
	\item Heeft u met Prince2 of PMBOK gewerkt?
	
		{\it Ik heb met Prince2 gewerkt.}
	\begin{itemize}
		\item Welk Prince2 certificaat heeft u behaald?
		
		{\it Het eerste certificaat, practitioners}
		
		\item Hoeveel jaar geleden was dit?
		
		{\it 
			Dit was 10 jaar geleden.
		}
	
		\item Was dit al binnen de Cronos group?
		
		{\it 
			Neen, dat was bij Fortis
		}
	
		\item Welke rol had u binnen het project management team?
		
		{\it 
			Ik ben lang business analyst geweest.
		}
	\end{itemize}
	\item Waarom heeft u voor een Prince2 certificaat gekozen?
	
	{\it 
		Ik was verplicht, Fortis werkte volledig via PRINCE2.
		
		Bij ING was dit eigenlijk ook hetzelfde, dit is een vrij universeel gegeven bij de Belgische banken.
	}
	\item Werkte u toen al Agile?
	
	{\it 
		Vroeger niet, nu wel.
		
		Bij Fortis was het altijd waterfall. Het grootste nadeel was dat er constant iemand met administratie bezig was.
		
		Dit is ook de reden dat ik gestopt ben bij Fortis en naar een kleiner bedrijf ben gegaan die software maakt voor externen. Daar was het wel Agile, en ik heb een grote voorkeur voor Agile werken.
	}
	\item Wat vond u van de PRINCE2 methodologie?
	
	{\it 
		Ik vond het goed dat je moest nadenken welke business vraag je gaat oplossen met je project.
		
		Dankzij het waterfall principe was je al makkelijk een half jaar verder, zelf als er enkel kleine aanpassingen moesten gebeuren.
		
		Een combinatie tussen PRINCE2 en Agile lijkt me wel zeer goed.
	}
	\item Nu bestaat er een PRINCE2 Agile certificaat, wat vind u hiervan?
	
	{\it 
		Ik vind het zeer goed dat Agile begint samen te komen met PRINCE2.
	}
	\item Vanaf welk punt denkt u dat het best is om van PRINCE2 naar een meer Agile werkmethode over te schakelen?
	
	{\it 
		Eerst de project brief opstellen. Daarna zou je eigenlijk al direct Agile kunnen werken. Het sturings comitee van PRINCE2 mis ik wel beetje bij Agile. 
		
		De PIDs (Project Initiation Document) niet te groot maken en dan beginnen met sprints, zeker af en toe eens terugkomen op de PID.
	}
	\begin{itemize}
		\item Wat zou volgens u het ideale tijdsschema zijn om de project brief te reviewen?
		
		{\it 
			Als een sprint gemiddeld 2 weken is, zou je om de 5 sprints eens terug moeten keren naar de project brief.
		}
	\end{itemize}
	\item Welke items van PRINCE2 mist u binnen Agile?
	
	{\it 
		Het initiele nadenken is zeer handig om te doen. Ook is het sturings comitee zeer goed, of gewoon iemand die praat vanuit het perspectief van de klant is al goed.
		
		Documentatie van PRINCE2 is ook zeer goed, dat je niet regelmatig verkeerde uren inschat of te grote items maakt in de product backlog.
		
		Nu gebruiken we veel de T-shirt methode. Hierbij schatten we de taken in op grote in S, M, L en XL. 
		
		Natuurlijk is het niet de bedoeling om de volledige documentatie van PRINCE2 te gebruiken, maar beter nadenken over wat er nuttig is, is wel zeer belangrijk. Het bijleren van fouten gaat soms verloren bij Agile.
	}
	\item Welke moeilijkheden heeft u meegemaakt in de switch van PRINCE2 naar volledig Agile werken?
	
	{\it 
		Dit was zeer moeilijk. Het was totaal niet duidelijk hoe groot een item in de product backlog moest zijn, meestal was dit veel te groot ingeschat.
		
		Het opsplitsen van die items is ook zeer lastig geweest.
		
		De daily standups was even wennen, maar dit vind ik echt top.
		
		Agile vind ik veel aangenamer om in te werken.
	}
	 
	\begin{itemize}
		\item Welke aanpassingen in denkwijze heeft u ondervonden?
		
		{\it 
			In Agile is er sowieso een veel grotere werkdruk, elke sprint moet er iets opgeleverd worden. Vergeleken met waterfall waar je zonder problemen een week niets kon doen en niemand zou er iets van gemerkt hebben.
		}
	\end{itemize}
	\item Hoe denkt u dat PRINCE2 en Agile zouden kunnen samenwerken?
	
	{\it 
		Ik heb geen expertise in de samenwerking van beide principes, maar wel in elk principe appart.
	}
	\begin{itemize}
		\item Hoe zou u Prince2 en Agile combineren?
		
		{\it 
			Het combineren lijkt voor mij zeer moeilijk, zeker door de andere mentaliteit bij banken. 
			
			Project managers zijn helemaal andere mensen, ze zullen zich uitendelijk wel aanpassen. Maar dit zal zeer moeizaam gebeuren.
			
			Een plotse overstap is niet mogelijk.
	}
		\item Hoe zou men reageren 10 jaar geleden als er een verandering zou komen om Prince2 meer Agile te maken?
		
		{\it 
			Agile is ook niet voor iedereen, er moet gepresteerd worden.
			
			Er valt veel te leren.
		}
	\end{itemize}
\end{itemize}
\subsection{Analyse antwoorden}
Het gesprek met Jocelijn Geurts van de Cronos groep verliep heel vlot. Jammer genoeg had zij zelf nog nooit de combinatie van Agile en PRINCE2 samen gebruikt. Los daarvan bezit ze wel expertise binnen beide gebieden. Haar antwoorden waren grotendeels een bevestiging van de theorie, maar moeten door haar duidelijke voorkeur voor Agile soms wel kritisch bekeken worden. Ze bevestigde de administratieve aard van PRINCE2 en de iets lossere, maar intensievere aard van Agile. Doordat ze PRINCE2 en Agile nog niet gecombineerd had hebben we onze vragen een beetje moeten aanpassen. Desondanks geven de antwoorden ons wel een insight in ons onderwerp. \\\\
Uit onze literatuurstudie bleek reeds dat PRINCE2 en Agile werken zeker mogelijk is en dit wordt ook bevestigd in ons interview. Natuurlijk liggen ze wel ver uit elkaar en zal de switch niet gemakkelijk geweest zijn. Uit het interview blijkt ook dat de methodes die bedrijven gebruiken ook heel hard vast hangt aan wat voor soort bedrijf het is en de grootte van het bedrijf. Ieder bedrijf heeft zijn eigen motivatie en doelen waardoor ze een specifieke methode kiezen. Als je ze wil combineren is PRINCE2 Agile een ideale oplossing. Dit certificaat is natuurlijk geleidelijk aan gegroeid. \\\\
Uit de combinatie van onze literatuur studie en het interview kunnen we dus besluiten dat PRINCE2 en Agile zeker samen kan, maar niet zonder problemen.

\section{Reflectie}
Het verzamelen voor de literatuurstudie was geen probleem. PMBOK, PRINCE2 en Agile worden wereldwijd gebruikt en zijn dus heel goed gedocumenteerd. De info die we vonden op het internet en de info die we uit ons interview hebben gehaald bevestigen elkaar ook. Een bedrijf vinden dat openstond voor een interview daarentegen was een heel moeilijke opdracht. We hebben een hele waslijst van Belgische multinationals gecontacteerd (lijst in de bronvermelding), maar kregen maar zelden een antwoord. De antwoorden die we ontvingen waren dat ze niet geïnteresseerd waren of dat ze niemand hadden met de expertise om onze vragen te beantwoorden. De organisatietypes die werden voorgesteld waren grote banken, overheid of multinationals, maar van deze bedrijven was er heel weinig intentie om ons te helpen bij onze bedrijfscase. Gelukkig heeft Tijl Zwartjes via een Hackathon voor de Cronos Group nog een interview kunnen strikken.
\section{Bronvermelding}

\subsection{Gecontacteerde bedrijven}
\begin{itemize}
	\item ExxonMobil
	\item Anheuser-Busch Inbev NV / Ampar BVBA
	\item Ansell Healthcare Europe NV
	\item Astra Sweets NV
	\item Atlas Copco Airpower NV
	\item BASF Antwerpen NV
	\item Belgacom International Carrier Services NV
	\item BP Aromatics Limited NV
	\item Bridgestone Europe NV
	\item British American TobaccoCoordination Center VOF
	\item Capsugel Belgium NV
	\item Celio International NV/SA
	\item Chep Equipment Pooling NV
	\item Delta Light NV
	\item Dow Corning Europe NV/SA
	\item Esko-Graphics BVBA
	\item EVAL Europe NV
	\item Evonik Oxeno Antwerpen NV en “NewCo”
	\item Flir Systems Trading Belgium BVBA
	\item Henkel Electronic Materials (Belgium) NV
	\item Kinepolis Group NV
	\item Knauf Insulation SPRL
	\item LMS International NV
	\item Luciad NV
	\item Magnetrol International NV
	\item Mayekawa Europe NV
	\item Noble International Europe BVBA
	\item Nomacorc SA
	\item Omega Pharma International NV
	\item Ontex BVBA
	\item Pfizer Animal Health SA / Zoetis Belgium SA
	\item Puratos NV
	\item Soudal NV
	\item St. Jude Medical Coordination Center BVBA
	\item Tekelec International SPRL
	\item The Heating Company BVBA
	\item Trane BVBA
	\item VF Europe BVBA
	\item Victaulic Europe BVBA
	\item Wabco Europe BVBA
	\item Infrabel
	\item NMBS
	\item ING
	\item BNP Baripas Fortis
	\item AXA
	\item KBC
	\item Cronos
	
	
\end{itemize}

\subsection{Literatuurstudie bronnen}
\printbibliography
\end{document}
